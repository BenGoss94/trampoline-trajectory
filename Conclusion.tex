\section{Conclusion}\label{conclusion}


%results analysis
%does varying time affect the max height reached?
%does varying distance affect max height reached?
%is any of the findings more conclusive than the other?





 \noindent As discussed previously in Section \ref{trampsafety}, adding a protective netting to the trampoline will drastically improve safety by preventing users falling off. Since falling off was discovered to be the largest cause of injury, it is the best place to implement some safety measure, such as the netting enclosure. The decision on how high this netting will be required in order to prevent users falling off, given the trampoline size studied, can be made. For the three metre diameter trampoline modelled, the maximum height reached by a mass is 5 metres, as seen in Section \ref{results}. Therefore, assuming that this mass represents the centre of mass of an 11 year old child of height 1.44 metres \cite{childweight}, adding half the child's height to this will give the height the netting needs to be, 5.72 metres. \\

 
% state what the heights were and therefore what size nets to use (height of mass + some decided amount extra to replicate human)
 % actually see if the heights of the jump (what will be a little under the size of our nets) match up to real life heights



%lever effect
\noindent As seen from the results of Figures 13, 14 and 15, the first mass on the trampoline is the one that leaves with greatest height. The mechanism that appears to power this motion is a `lever' between the two masses. There is some pivoting point between the two masses that the inextensible string, that models the trampoline, passes through. The pivot motion acts through a vertical plane rather than a point, based on the vertical components of the two masses. The trampoline behaves like a rigid body between the masses and hence the larger the distance between them creates a larger `lever' able to project the first mass to a greater height.
%double check.
\\
\\
\noindent To conclude, the further away the masses are from each other, the higher the mass already situated on the trampoline will go. Also, when the mass that is already on the trampoline is at its lowest point, and the second mass hits the trampoline, a height increase for the first mass will occur. Utilising both of these together would cause the maximum height to be achieved from the mass of the trampoline. This, when applied to an adult and a child, and the child is the lowest, would cause the most unsafe situation with the largest height.  