\section{Introduction}

Trampolining is a widely appreciated pastime and sport across the world for many children and adults. Since the rise in popularity of household trampolines in the 1990's, the issue of trampolining related injuries has become more apparent. %expand with a little more background information possibly
\\
\\
\noindent When two individuals jump on a trampoline at approximately the same time it can result in one of the individuals reaching a greater height whilst the other individual bounces up to a lower height than before. This is the result of a transfer of energy between the two involved. The positions and weights of the two bouncers can lead to the individual that bounces highest being flung off the trampoline, possibly resulting in injury. 
\\
\\
\noindent This project aims to investigate when injuries can occur using a trampoline and what measures can be implemented to prevent them. In order to achieve this, the physics of a trampoline need to be understood and how this energy transfer discussed can increase the likelihood of injury in users. It is also necessary to determine if there are any safe ways to allow both persons to bounce that minimise risk of injury through one participant bouncing off the trampoline.
\\
\\
By creating a model, simulations of the trampoline can be run for two people and what happens to them in varying situations can be tested. This information can aid in recommending courses of action to improve the safety of trampolining, and also providing suitable guidelines for both adults and children on the trampoline.
%paragraph about use of the model.
