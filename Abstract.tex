%Intro
\noindent The aim of this project is to research the safety issues presented when two people simultaneously jump on a trampoline and when they are most likely to be in an unsafe situation. Safety issues that are related to falling off of the trampoline shall be discussed alongside the resulting unsafe situations to find possible measures that can be taken to improve the safety of trampolining. A trampoline works by converting the kinetic energy of the person falling on to the trampoline, and this energy is converted to potential energy. The potential energy is then converted back to kinetic energy as the person is propelled off of the trampoline. When two people are bouncing on a trampoline, it is possible for the one person to `absorb' the energy from the other person's bounce, thus propelling them even further in to the air, and the person whose energy has been `absorbed' has a lower bounce.
\\
\\
\noindent A model was created to find the most unsafe scenerio possible. This looked to find the combination of parameters which caused one of the masses to jump above an unsafe height. The two important factors looked at were the time lag between the two people and the distance between them when they are both jumping on the trampoline. 
\\
\\
\noindent The results produced by the model suggest that the most unsafe scenario appears when there is a large horizontal distance between the two people and when one person is at the bottom of their jump onto the trampoline when the second person lands on the trampoline.